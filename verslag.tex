
\documentclass[10pt]{article}

\parindent=0pt

\usepackage{fullpage}

\frenchspacing

\usepackage{microtype}

\usepackage[english,dutch]{babel}

\usepackage{graphicx}

\usepackage{listings}

\lstset{language=C++, showstringspaces=false, basicstyle=\small,
  numbers=left, numberstyle=\tiny, numberfirstline=false, breaklines=true,
  stepnumber=1, tabsize=8, 
  commentstyle=\ttfamily, identifierstyle=\ttfamily,
  stringstyle=\itshape}

\usepackage[setpagesize=false,colorlinks=true,linkcolor=red,urlcolor=blue,pdftitle={Kunstmatige Intelligentie bij Tetris},pdfauthor={Alex Keizer}]{hyperref}

\author{Bram Honig \and Alex Keizer}
\title{Kunstmatige Intelligentie bij Tetris}

\begin{document}

\selectlanguage{dutch}

\maketitle

\section{Inleiding} 
Tetris staat voornamelijk bekend als een spelletje waar je een goed reactievermogen voor moet hebben. Toch zorgen de vele mogelijke manieren om een stuk te plaatsen het feit dat je niet weet welke stukken nog gaan komen ervoor dat het zeker niet recht toe recht aan duidelijk is wat de beste zet op een gegeven moment is. In onze zoektocht zullen wij ons slechts beperken tot het stuk wat we op een gegeven moment ktijgen en niet nog zetten vooruit gaan denken.

Waarschijnlijk kennen de meesten de regels van Tetris wel, maar voor de volledigheid zullen we ze nog noemen. Elke zet krijg je een stuk, waar de speler moet bepalen op welke plek en met welke rotatie we het laten vallen. Dit mag, in onze variatie, tijdens het vallen niet veranderd worden. Als een rij compleet vol zit zal deze worden weggehaald en de vakjes erboven zullen een rij naar beneden vallen. Het spel is voorbij wanneer een stuk niet meer op het bord past.

\section{Uitleg probleem}\label{probleem}

Elke ``beurt'' krijgen we een (willekeurig) stuk, onze zet bestaat dan vervolgens uit de kolom waar we dat stuk laten vallen en met welke rotatie we dat doen. In onze variatie hebben we dus geen kennis over welk stuk gaat volgen. Aangezien er aanzienlijk wat mogelijkheden zijn is het erg lastig om te bepalen wat de optimaalste zet is. Zelfs als een preciese reeks stukken gegeven wordt is het een NP-hard probleem om de beste zetten te vinden. \cite{nphard} (zie sectie \ref{relevant})

\section{Relevant werk}\label{relevant}

In eerder onderzoek is aangetoond dat het bepalen van de beste zetten een NP-hard probleem\cite{nphard} is---dit geldt zowel als je geplaatste stukken of weggehaalde rijen als score gebruikt. Dit houdt in dat als er een oplossing wordt aangeboden deze heel makkelijk te verifieren is, echter om een (optimale) oplossing te vinden kost dat op zijn minst polynomiale tijd. Als het probleem complexer wordt, wordt de tijd die het kost om een oplossing te vinden extreem groeit.

Om deze reden is er ook onderzoek gedaan naar hoe goed een heuristisch algoritme tetris kan spelen, waarbij een zet a.d.v. een aantal kenmerken en co\"{e}ficenten voor die kenmerken worden beoordeeld. Bij het onderzoek uit 2015 ``An Evolutionary Approach to Tetris'' \cite{genetic} werden 12 kenmerken gebruikt waar de evualatie functie zich op baseerde. Drie hiervan gebruiken wij ook: (zie sectie \ref{aanpak} voor uitleg van deze punten)
\begin{itemize}
\item Hoogste rij
\item Weggehaalde rijen
\item Gaten
\item Hoogte verschil
\end{itemize}

Om het gewicht van elk kenmerk te bepalen werd gebruik gemaakt van een genetisch algoritme. Een van de conclusies was dat er veel locale optima waren waar het algoritme niet kon ontsnappen. Verder werd geconstateerd dat het bepalen van de optimale gewichten erg veel tijd in beslag neemt---een van de tests heeft 3 maanden gedraaid.


\section{Aanpak}\label{aanpak}
Als eerste poging is er een Monte Carlo algoritme ge\"{i}mplementeerd. Bij elke beurt wordt elke mogelijke zet een vast aantal willekeurige potjes gespeeld. De zet waarbij in al die spelletjes het meeste stukken werden gespeeld wordt gekozen als de beste zet.

Willekeurig spelen is niet erg effici\"{e}nt, dus hebben we ook een ``slim'' algoritme. Zoals gezegd in sectie \ref{probleem} wordt een speelboom al snel erg groot, wij hebben ons dus beperkt tot alle mogelijkheden met het huidige stuk. Elk van deze mogelijkheden wordt ge\"{e}valueerd aan de hand van de volgende punten:
\begin{itemize}

\item \textbf{Hoogste Rij} De hoogste rij met bezette vakjes (in het kwadraat)

\item \textbf{Hoogste Legen} Aantal lege vakjes in de hoogst bezette rij

\item \textbf{Weggehaalde rijen} Deze spreekt voor zich

\item \textbf{Gaten} Lege vakjes met erboven (in dezelfde kolom) ene niet-leeg vakje

\item \textbf{Hoogte Verschil} Het hoogteverschil tussen opeenvolgende rijen

\end{itemize}
Hoe hoger de stapel wordt, hoe erger het is als hij verder stijgt (je komt immers telkens dichter bij het einde). We hebben dus niet direct het nummer van de hoogste rij gepakt, maar het kwadraat daarvan als meetpunt. Elk van deze punten wordt vervolgens met een coeffic\"{i}ent vermenigvuldigt en bij elkaar opgeteld, dit is de uiteindelijke score voor dat bord (en de zet die tot dat bord heeft geleid). Verder wordt natuurlijk weer de zet met de hoogste score gedaan.

\section{Implementatie}

Wij hebben in C++ geprogrammeerd. De basis voor de slimme-, en Monte-Carlo speler is hetzelfde. Met \verb+possibilities (piece)+ wordt bepaald hoeveel zetten er mogelijk zijn, elke zet wordt vervolgens gescoord door ofwel \verb+evaluatemove(piece, themove)+ (slim), ofwel \verb+evaluateMonteCarlo(piece, themove)+ (Monte Carlo).

\verb+evaluatemove(piece, themove)+ maakt als eerste een kopie van het \verb+Tetris+ object, voert de zet uit en bepaalt de eerder gegeven kenmerken---hoogste rij, hoogste legen, weggehaalde rijen, gaten en hoogte verschil---deze worden vermenigvuldigd met de bijbehorende co\"{e}fficienten (vastgelegd in het \verb+TetrisScore+ struct gegeven in de constructor van \verb+Tetris+) en opgeteld. Deze laatste waarde is wat gereturned wordt als score.
\verb+evaluatemontecarlo(piece, themove)+ maakt eveneenst eerste een kopie en voert de zet uit. Verder worden er een vast aantal ($monte\_loop$) keer nog een kopie van de staat na de zet gemaakt, een willekeurige potje gespeeld met \verb+playrandomgame( )+ en de uiteindelijk verkregen $piececount$, het aantal gespeelde stukken, van alle potjes opgeteld. Deze totaalscore wordt gereturned.

Welke van de evaluatiefuncties gebruikt wordt, wordt bepaald door de $EvalName$ die aan \verb+playgame(EvalName eval)+ gegeven wordt.

\section{Experimenten}

Ten eerste hebben wij bij de Monte Carlo speler het aantal willekeurige potjes---$monte\_loop$---gevarieerd. Wij hebben voor elke variatie 1000 keer gespeeld, waarbij we bijhouden hoe lang dit duurde en hoe veel stukken gespeeld zijn. Deze tests zijn gedaan op een $15 \times 20$ bord. Absoluut is deze tijd sterk afhankelijk van o.a. de gebruikte hardware en niet zo interessant, maar de stijging geeft wel degelijk informatie. De tijd uitgezet tegenover $monte\_loop$ is bijna lineair (zie Figuur \ref{fig:monte_score}). Het meeste tijd zal worden gespendeerd aan \verb+playrandomgame( )+, die in \'e\'en spel $\#zetten \times monte\_loop$ keer wordt aangeroepen. Als $monte\_loop$ groter wordt, zal het algoritme iets beter werken en dus meer zetten doen. Echter is $monte_loop$ vaak significant groter, waardoor het alsnog bijna lineair is. De behaalde score tegenover $monte\_loop$ is exponenti\"el dalend. Door de willekeurige aard van het Monte Carlo algoritme is dit niet heel vreemd, een $11^e$ poging heeft redelijk kans een betere weg te vinden dan de eerste $10$. Een $101^{ste}$ poging is daarentgen veel minder waarschijnlijk beter dan de eerdere $100$.


We hebben ook de score per tijdseenheid tegenover $monte\_loop$ geplot in Figuur \ref{fig:monte_eff}. Hier is duidelijk te zien wat ook al uit de voorgaande analyse geconcludeerd kan worden, hoe groter $monte\_loop$ wordt, hoe minder effici\"ent het algoritme wordt. Elke stijging van $monte_loop$ zorgt voor een (iets) grotere stijging van de gespendeerde tijd, meer steeds minder stijging van behaalde score.

[ INSERT MONTE CARLO FIGUREN ]

We hebben eveneens beide spelers getest op verschillende bord groottes. Voor Monte Carlo hebben we $monte\_loop = 500$ gebruikt, voor de slimme speler hebben we de co\"efficienten uit Figuur \ref{fig:smart_coef} gebruikt. De score is bepaald als gemiddelde van 500 gespeelde potjes, daarnaast hebben we de standaardfout van dat gemiddelde bepaald.

\begin{figure}
$$\begin{array}{l||c|c|c|c|c}
	    & 4 			& 8 			   & 12 			& 16 			 & 20 \\ 
	\hline\hline
	8  & 18.2 (\pm 11.5) 	& 24.3 (\pm 11.7) & 25.8 (\pm 11.1)  & 27.3 (\pm 9.6)  & 28.5 (\pm 7.8) \\
	12 & 41.7 (\pm 22.9) & 61.6 (\pm 28)    & 62 (\pm 23.5)     & 60.6 (\pm 18.2) & 59.1 (\pm 15.1)  \\
	15 & 59.7 (\pm 30.2) & 91.6 (\pm 41.8) & 86.7 (\pm 29)     & 81.4 (\pm 22..4) & 77.5 (\pm 16.4) \\
\end{array}$$
\caption{Gemiddeld behaalde zetten (en standaardfout) van Monte Carlo op verschillende bordgroottes, bepaald door 500 tetris potjes met $monte\_loop=500$}
\end{figure}

\begin{figure}
$$\begin{array}{l||c|c|c|c|c}
	    & 4 			& 8 			   & 12 			& 16 			 & 20 \\ 
	\hline\hline
	8  & 18.2 (\pm 11.5) 	& 24.3 (\pm 11.7) & 25.8 (\pm 11.1)  & 27.3 (\pm 9.6)  & 28.5 (\pm 7.8) \\
	12 & 41.7 (\pm 22.9) & 61.6 (\pm 28)    & 62 (\pm 23.5)     & 60.6 (\pm 18.2) & 59.1 (\pm 15.1)  \\
	15 & 59.7 (\pm 30.2) & 91.6 (\pm 41.8) & 86.7 (\pm 29)     & 81.4 (\pm 22..4) & 77.5 (\pm 16.4) \\
\end{array}$$
\caption{Gemiddeld behaalde zetten (en standaardfout) van de slimme speler op verschillende bordgroottes, bepaald door 500 tetris potjes met co\"efficienten van Figuur \ref{fig:smart_coef}}
\end{figure}

\section{Conclusie}

Leuk onderzoek, veelbelovend ook. Het ging helaas 
fout als de testopstelling niet verlicht was.
In de toekomst doen we dat anders.

\begin{thebibliography}{XX}

\bibitem{nphard}
Erik D. Demaine, Susan Hohenberger and David Liben-Nowell, Tetris is
Hard, Even to Approximate. In Proceedings of the 9th International Computing and
Combinatorics Conference (COCOON 2003) (2003).

\bibitem{genetic}
Niko B\"{o}hm Gabriella K\´{o}kai Stefan Mandl,
An Evolutionary Approach to Tetris
MIC2005: The Sixth Metaheuristics International Conference

https://www2.informatik.uni-erlangen.de/EN/publication/download/mic.pdf

\end{thebibliography}

\section*{Appendix: Code}

Er is gebruik gemaakt van de \href{http://www.liacs.leidenuniv.nl/~kosterswa/AI/iets.cc}{\underline{skeletcode}} die te vinden is via
de website van het college.
De code van het programma is als volgt:

\smallskip


\end{document}
